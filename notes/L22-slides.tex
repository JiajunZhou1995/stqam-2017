\documentclass{beamer}

\usetheme{Boadilla}

%\includeonlyframes{current}

\usepackage{times}
\usefonttheme{structurebold}
\usepackage{listings}
\usepackage{ragged2e}

\usepackage{pgf}
\usepackage{tikz}
\usepackage{alltt}
\usepackage[normalem]{ulem}
\usetikzlibrary{arrows}
\usetikzlibrary{automata}
\usetikzlibrary{shapes}
\usepackage{amsmath,amssymb}
\usepackage{rotating}
\usepackage{ulem}
\usepackage{listings}
\usepackage{enumerate}
\usepackage{tikz}
\tikzset{
  every overlay node/.style={
    draw=black,fill=white,rounded corners,anchor=north west,
  },
}
\def\tikzoverlay{%
   \tikz[baseline,overlay]\node[every overlay node]
}%

%\setbeamercovered{dynamic}
\setbeamertemplate{footline}[page number]{}
\setbeamertemplate{navigation symbols}{}
\usefonttheme{structurebold}

\title{Software Testing, Quality Assurance \& Maintenance---Lecture 22}
\author{Patrick Lam}
\date{March 1, 2017}

\colorlet{redshaded}{red!25!bg}
\colorlet{shaded}{black!25!bg}
\colorlet{shadedshaded}{black!10!bg}
\colorlet{blackshaded}{black!40!bg}

\colorlet{darkred}{red!80!black}
\colorlet{darkblue}{blue!80!black}
\colorlet{darkgreen}{green!80!black}

\newcommand{\rot}[1]{\rotatebox{90}{\mbox{#1}}}
\newcommand{\gray}[1]{\mbox{#1}}

\newenvironment{changemargin}[1]{% 
  \begin{list}{}{% 
    \setlength{\topsep}{0pt}% 
    \setlength{\leftmargin}{#1}% 
    \setlength{\rightmargin}{1em}
    \setlength{\listparindent}{\parindent}% 
    \setlength{\itemindent}{\parindent}% 
    \setlength{\parsep}{\parskip}% 
  }% 
  \item[]}{\end{list}}


\lstset{ %
language=C++,
basicstyle=\ttfamily,commentstyle=\scriptsize\itshape,showstringspaces=false,breaklines=true}

\begin{document}

\begin{frame}
  \titlepage
\end{frame}

\begin{frame}
  \frametitle{Last Time}

  \begin{changemargin}{2cm}
    MAY-beliefs versus MUST-beliefs\\[1em]
    Cross-checking beliefs
  \end{changemargin}
\end{frame}

\begin{frame}
\frametitle{Today}
\begin{changemargin}{2cm}
Inferring beliefs via statistics\\
\end{changemargin}
\end{frame}

\part{Inferring beliefs}
\frame{\partpage}

\begin{frame}[fragile]
\frametitle{Redundancy Checking}
  \begin{changemargin}{2cm}
    Assumption: code ought to do something\\[1em]
    Look for identity operations, e.g.\\
\hspace*{4em} {\tt x = x}, ~ {\tt 1 * y}, ~ {\tt x \& x}, ~ {\tt x | x}.\\
  \end{changemargin}
{\small
\begin{lstlisting}[language=C]
    /* 2.4.5-ac8/net/appletalk/aarp.c */
    da.s_node = sa.s_node;
    da.s_net = da.s_net;
\end{lstlisting}
}
  \begin{changemargin}{2cm}
    Also look for unread writes:\\
  \end{changemargin}

{\small \begin{lstlisting}[language=C]
    for (entry=priv->lec_arp_tables[i]; 
         entry != NULL; entry=next) {
      next = entry->next; // never read!
      ...
    }
\end{lstlisting} 
}
  \begin{changemargin}{2cm}
Redundancy suggests conceptual confusion.\\
\small \hfill (examples courtesy Dawson Engler)
 
  \end{changemargin}

\end{frame}

\begin{frame}[fragile]
\frametitle{From MUST to MAY}
  \begin{changemargin}{2cm}
    Preceding examples were about MUST beliefs:\\
\hspace*{2em} violations were clearly wrong.\\[1em]
    Let's examine MAY beliefs next:\\
    \begin{itemize}
      \item need more evidence of wrongdoing.
    \end{itemize}
  \end{changemargin}
\end{frame}

\begin{frame}
\frametitle{Verifying MAY beliefs}
  \begin{changemargin}{1.5cm}
\begin{enumerate}
\item    Record every successful MAY-belief check as ``check''.\\[1em]
\item    Record every unsucessful belief check as ``error''.\\[1em]
\item    Rank errors based on ``check'' : ``error'' ratio.
\end{enumerate}~\\[1.5em]
    Most likely errors: ``check'' is large, ``error'' small.
  \end{changemargin}
\end{frame}

\begin{frame}[fragile]
\frametitle{Let's find some MAY beliefs}
  \begin{changemargin}{2cm}
    use-after-free: 
 \begin{lstlisting}[language=C]
    free(p);
    print(*p);
 \end{lstlisting}
~\\[1em]
That is a MUST-belief. \\[1em]
However, other resources are freed by custom (undocumented) free functions.\\[1em]
Let's derive them behaviourally.
  \end{changemargin}
\end{frame}


\begin{frame}
\frametitle{Finding custom free functions}
  \begin{changemargin}{2cm}
    Key idea:\\
    If pointer {\tt p} not used after calling {\tt foo(p)},\\
    then derive a MAY belief that {\tt foo(p)} frees {\tt p}.\\[2em]

    Just assume all functions free all arguments.
\begin{itemize}
\item emit ``check'' at every call site;
\item emit ``error'' at every use.
\end{itemize}
    (in reality, filter functions with suggestive names).
  \end{changemargin}
\end{frame}

\begin{frame}
\frametitle{Example: finding free functions}
  \begin{changemargin}{2cm}
Putting that into practice, \\
we might observe:\\[1em]
\begin{tabular}{l|l|l|l|l|l}
\begin{minipage}{3em}
foo(p)
*p = x;
\end{minipage} &
\begin{minipage}{3em}
foo(p)
*p = x;
\end{minipage} &
\begin{minipage}{3em}
foo(p)
*p = x;
\end{minipage} &
\begin{minipage}{3em}
bar(p)
p = 0;
\end{minipage} &
\begin{minipage}{3em}
bar(p)
p=0;
\end{minipage} &
\begin{minipage}{3em}
bar(p)
*p = x;
\end{minipage} 
\end{tabular}
~\\[1em]
Rank {\tt bar}'s error first.\\
Sample results: 23 free errors, 11 false positives.
  \end{changemargin}
\end{frame}

\begin{frame}
\frametitle{More statistical techniques: nullness checks}
  \begin{changemargin}{2cm}
    Situation: \\[1em]
    \hspace*{1em} Want to know which routines may return {\tt NULL}.\\[1em]
    Possible solution: static analysis to find out.\\[1em]
    Problems:
    \begin{itemize}
      \item difficult to know statically (``{\tt return p->next;}''?)
      \item get false positives: \\ functions return {\tt NULL} under special cases only.
    \end{itemize}
    

  \end{changemargin}
\end{frame}

\begin{frame}
\frametitle{Applying a statistical technique to nullness checks}
  \begin{changemargin}{2cm}
    Instead: let's observe what the programmer does.\\
    Again, rank errors based on checks vs non-checks.\\[2em]
    Just assume {\bf all} functions can return {\tt NULL}.
\begin{itemize}
  \item pointer checked before use: emit ``check'';
  \item pointer used before check: emit ``error''.
\end{itemize}
  \end{changemargin}
\end{frame}

\begin{frame}
\frametitle{Example: finding NULL-returning functions}
  \begin{changemargin}{2cm}
This time, 
we might observe:\\[1em]
  \end{changemargin}
\begin{tabular}{l|l|l|l}
\begin{minipage}{5em}
p = bar(...);\\
*p = x;
\end{minipage} &
\begin{minipage}{7em}
p = bar(...);\\
if (!p) return;\\
*p = x;
\end{minipage} &\begin{minipage}{7em}
p = bar(...);\\
if (!p) return;\\
*p = x;
\end{minipage} &\begin{minipage}{7em}
p = bar(...);\\
if (!p) return;\\
*p = x;
\end{minipage}
\end{tabular}\\[1em]
  \begin{changemargin}{2cm}
Sort errors based on ``check'':``error'' ratio.\\
Sample results: 152 free errors, 16 false positives.
  \end{changemargin}
\end{frame}

\begin{frame}
\frametitle{General statistical technique}
  \begin{changemargin}{2cm}
``a(); \ldots b();'' implies MAY-belief that a() followed by b().\\
(is it real or fantasy? we don't know!)\\[2em]
Algorithm: 
\begin{itemize}
\item assume every {\tt a}--{\tt b} is a valid pair;
\item emit ``check'' for each path with ``a()'' and then ``b()'';
\item emit ``error'' for each path with ``a()'' and no ``b()''.
\end{itemize}
(actually, prefilter functions that look paired).
  \end{changemargin}
\end{frame}

\begin{frame}
\frametitle{Example: general technique}
  \begin{changemargin}{2cm}
Consider:\\[1em]
  \end{changemargin}
\begin{tabular}{l|l|l}
\begin{minipage}{10em}
foo(p, \ldots);\\
bar(p, \ldots); // check 
\end{minipage} &
\begin{minipage}{10em}
foo(p, \ldots);\\
bar(p, \ldots); // check 
\end{minipage} &
\begin{minipage}{12em}
foo(p, \ldots);\\
// error: foo, no bar!
\end{minipage}
\end{tabular}
\end{frame}

\begin{frame}[fragile]
\frametitle{Application: course project}
\begin{tabular}{ll}
\begin{minipage}{10em}
\scriptsize
\begin{lstlisting}
void scope1() {
 A(); B(); C(); D();
}

void scope2() {
 A(); C(); D();
}

void scope3() {
 A(); B();
}

void scope4() {
 B(); D(); scope1();
}

void scope5() {
 B(); D(); A();
}

void scope6() {
 B(); D();
}
\end{lstlisting}
\end{minipage} &\begin{minipage}{18em}
``A() and B() must be paired'':\\
either A() then B() or B() then A().\\[2em]
\begin{tabbing}
{\bf Support} =  \= \# times a pair of functions \\
\> appears together.\\
\> \hspace*{2em} support(\{A,B\})=3
\end{tabbing}
~\\[1em]
{\bf Confidence(\{A,B\},\{A\}}) = \\ \hspace*{2em} support(\{A,B\})/support(\{A\}) = 3/4
\end{minipage}
\end{tabular}
\end{frame}

\begin{frame}[fragile]
\frametitle{Application: course project}
\begin{tabular}{ll}
\begin{minipage}{10em}
\scriptsize
\begin{lstlisting}
void scope1() {
 A(); B(); C(); D();
}

void scope2() {
 A(); C(); D();
}

void scope3() {
 A(); B();
}

void scope4() {
 B(); D(); scope1();
}

void scope5() {
 B(); D(); A();
}

void scope6() {
 B(); D();
}
\end{lstlisting}
\end{minipage} &\begin{minipage}{18em}
Sample output for support threshold~3, confidence threshold 65\% (intra-procedural analysis):
\small
\begin{itemize}
\item bug:A in scope2, pair: (A B), support:~3, confidence: 75.00\%
\item bug:A in scope3, pair: (A D), support:~3, confidence: 75.00\%
\item bug:B in scope3, pair: (B D), support:~4, confidence: 80.00\%
\item bug:D in scope2, pair: (B D), support:~4, confidence: 80.00\%
\end{itemize}
\end{minipage}
\end{tabular}
\end{frame}

\begin{frame}[fragile]
  \frametitle{Why are we doing this again?}
    \begin{lstlisting}[language=C,commentstyle={\color{red}\bf}]
/* 2.4.0:drivers/sound/cmpci.c:cm_midi_release: */
lock_kernel(); // [PL: GRAB THE LOCK]
if (file->f_mode & FMODE_WRITE) {
  add_wait_queue(&s->midi.owait, &wait);
  ...
  if (file->f_flags & O_NONBLOCK) {
    remove_wait_queue(&s->midi.owait, &wait);
    set_current_state(TASK_RUNNING);
    return -EBUSY; // [PL: OH NOES!!1]
  }
  ...
}
unlock_kernel();
    \end{lstlisting}
    \begin{changemargin}{1cm}
      \Large
      Problem: lock() and unlock() must be paired!
  \end{changemargin}
\end{frame}

\begin{frame}
  \frametitle{Summary: Belief Analysis}
    \begin{changemargin}{2cm}
      We don't know what the right spec is.\\
      Instead, look for contradictions.\\[2em]
      MUST-beliefs: contradictions = errors!\\[1em]
      MAY-beliefs: pretend they're MUST, rank by confidence.\\[2em]
      (Key assumption: most of the code is correct.)
    \end{changemargin}
\end{frame}

\begin{frame}
  \frametitle{Further references}

Dawson~R. Engler, David Yu Chen, Seth Hallem, Andy Chou and Benjamin Chelf.\\
``Bugs as Deviant Behaviors: A general approach to inferring errors in systems code''.\\
In SOSP '01.\\[1em]

Dawson~R. Engler, Benjamin Chelf, Andy Chou, and Seth Hallem.\\
``Checking system rules using system-specific, programmer-written
  compiler extensions''.\\
In OSDI '00 (best paper).\\
\url{www.stanford.edu/~engler/mc-osdi.pdf}\\[1em]

Junfeng Yang, Can Sar and Dawson Engler.\\
``eXplode: a Lightweight, General system for Finding Serious Storage System Errors''.\\
In OSDI'06.\\
\url{www.stanford.edu/~engler/explode-osdi06.pdf}
\end{frame}


\end{document}
