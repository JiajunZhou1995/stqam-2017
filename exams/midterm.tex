\documentclass[12pt]{article}
\usepackage{graphicx}
\usepackage{listings}
\usepackage{fullpage}
\usepackage{tikz}
\usepackage{enumitem}
\usetikzlibrary{shapes,arrows,calc,automata}

\lstset{ %
language=Java,
basicstyle=\small \ttfamily,commentstyle=\scriptsize\itshape,showstringspaces=false,breaklines=true,numbers=left}

\usepackage{fontspec}
\setmonofont{Cousine}[Scale=MatchLowercase]

\begin{document}

\title{Software Testing, Quality Assurance \& Maintenance (ECE453/CS447/SE465): Midterm}
\author{}
\renewcommand{\today}{}
\maketitle

 ~\\[-7em]

\begin{center}
{\Large February 17, 2017}
\end{center}

This open-book midterm has 5 questions and 80 points. Answer the
questions in your answer book. You may consult any printed material
(books, notes, etc).


\section*{Question 1: Test Design (10 points)}
I've included below some test code from bukkit's {\tt
  BukkitMirrorTest}.  Split this test into well-designed independent
subtests. I recommend 5 tests; use your judgment to decide what the
subtests should be. Write out the valid JUnit test cases after
splitting. You may write e.g. ``// L4'' to indicate that you are
including line 4 from the provided code. Write a sentence explaining
the benefits of splitting the test.

\begin{lstlisting}[basicstyle=\scriptsize \ttfamily]
  @Test
  public void bunchOfTests() throws Throwable {
      assertThat(Modifier.isStatic(bukkit.getModifiers()), is(true));
      assertThat(bukkit.isAnnotationPresent(Deprecated.class), is(server.isAnnotationPresent(Deprecated.class)));
      assertThat(bukkit.getReturnType(), is((Object) server.getReturnType()));
      assertThat(bukkit.getGenericReturnType(), is(server.getGenericReturnType()));
      assertThat(bukkit.getGenericParameterTypes(), is(server.getGenericParameterTypes()));
      assertThat(bukkit.getGenericExceptionTypes(), is(server.getGenericExceptionTypes()));
  }
\end{lstlisting}

\vspace*{-1em}
\section*{Question 2: Fuzzing (10 points)}
% I provide code and a test case generator. Test cases all get filtered out at a bottleneck in the code. Provide a test case generator that explores more of the code. Explain the phenomenon.
\begin{lstlisting}
  float computeAngle(int x, int y, int z) {
    if (x * x + y * y != z * z) {
      throw new IllegalArgumentException("bad triangle");
    }
    return Math.atan2(y, x);
  }
\end{lstlisting}
\begin{itemize}[noitemsep]
\item (5 points) Write a Java function {\tt naiveComputeAngleTest()} that generates a completely-pseudorandom input triple $(x, y, z)$ for this method and calls it. Would {\tt naiveComputeAngleTest()} give you good insight
  into what {\tt computeAngle()} does? Explain your answer.
\item (5 points) Write a Java function {\tt smartComputeAngleTest()} that randomly generates an input that never triggers an {\tt IllegalArgumentException}. You may use {\tt Math.sqrt()}, which returns a double.
\end{itemize}
(To get a pseudorandom number: {\tt Random r = new Random(); i = r.nextInt(); })

\newpage
\section*{Question 3: Short-circuit evaluation (15 points)}
Here is a simplified version of code from Assignment 1.
\begin{lstlisting}
  String buildCommand(String formatString) {
    // returns the index at which '$' appears in formatString or -1 if not present
    int index = formatString.indexOf("$");
    while (index != -1) {
      if (index > 0 && formatString.charAt(index - 1) == '!') {
        // removes the '$' at index from formatString
        formatString = formatString.substring(0, index - 1) + formatString.substring(index);
        index = formatString.indexOf("$", index);
        continue;
      } else {
        ;
      }
      index = formatString.indexOf("$", index + 1);
    }
  }
\end{lstlisting}
\begin{itemize}[noitemsep]
\item (5 points) Draw a control-flow graph for this method. Nodes as basic blocks, labelled with line
  numbers. Explicitly write boolean conditions. (Be careful with the short-circuit evaluation.) I had 7 nodes.
\item (10 points) Test suite \verb+"$"+, \verb+"!$"+ achieves 100\%
  statement coverage but jacoco says ``1 branches out of 4 missed'' at
  the {\tt if} statement. Describe the missing branch and explain why
  it is missing (for 5 points) and write down a test case that achieves 100\%
  branch coverage (another 5 points). (Warning: because I removed code, this behaves
  slightly differently than on the assignment).
\end{itemize}

\newpage
\section*{Question 4: Statement and Branch Coverage (20 points)}
\begin{lstlisting}
public static List<String> wrap(String input, int line_length) {
  List<String> rv = new ArrayList<String>();
  int last_break = -1, last_space = 0;
  
  for (int i = 0;
       i < input.length();
       i++) {
    if (input.charAt(i) == ' ') {
      last_space = i;
    }
    
    if (i - last_break > line_length) {
      rv.add(input.substring(last_break + 1, last_space));
      last_break = i;
    }
  }
  if (last_space >= last_break + 1) {
    rv.add(input.substring(last_break + 1, last_space));
  }
  return rv;
}
\end{lstlisting}

\vspace*{1em} \noindent
(a) (5 points) Draw a minimal-node (9) control-flow graph for this method, with
nodes as basic blocks. Label the nodes with the line numbers on the left.

\vspace*{1em} \noindent
(b) (10 points) Consider \verb+input = "one three two ", line_length = 9+.
Argue that this input achieves statement coverage. You may use the fact that the output is:
\begin{verbatim}
$ java Justify 9 "one three two "
one three 
two
\end{verbatim}

\vspace*{1em} \noindent
(c) (5 points) Does my input also achieve branch coverage? Why or why not? (If so, say why; if not,
say specifically which branch is not covered).

\newpage
\section*{Question 5: Understanding Mutation (25 points)}
(a) (10 points) Write down a non-trivial mutant of {\tt wrap()} that
is killed by {\bf my} test case (from (1b)). (Write the line number that you are changing
and what you want it to say, e.g.:)

\verb+Line 5: for (int i = -1;+

\noindent
Explain why my test case
kills the mutant (show outputs). Also show why the mutant is non-trivial; that is,
find a non-empty input on which the mutant and the original do not differ.
% last_break = 0 should work if you give it a one-line input for non-triviality

\vspace*{1em} \noindent
(b) (5 points) Write down a test case for {\tt wrap()} (from Question
4) that reveals a bug in the original program. Include actual and expected output.
(Contract: the concatenation of the strings
returned from {\tt wrap()} must equal {\tt input}.)
You do not have to identify or fix the bug.
  Clearly, this
test case does not improve statement coverage beyond the input from
(4b).

\vspace*{1em} \noindent
(c) (10 points) Write down a different non-trivial mutant of {\tt
  wrap()} that is killed by {\bf your} test case (from part b) and explain why your test
case kills the mutant (show outputs).  Again explain why the mutant is non-trivial;
that is, find a non-trivial input on which the mutant and the original
do not differ.

Observe that you've used mutation to improve your test suite from what
it was in (4b).  You generated a mutant, and the test case from (4b)
killed the mutant from (4c). On the other hand, both statement and branch
coverage don't get you anything more.

%[or, provided bug not killable by mutation]

%does your test case improve stmt coverage? branch coverage?


%\section*{Question 6: Finite State Machines (5 points)}
%read a document describing a system; create an FSM and test cases for that system

\end{document}
